\documentclass{book}
\usepackage[spanish]{babel}
\usepackage[utf8]{inputenc}
\usepackage[colorlinks = true, linkcolor = magenta]{hyperref}
\usepackage{graphicx}

\begin{document}

	\begin{titlepage}
		\centering
		{\bfseries\LARGE Universidad de La Habana \par }
		\vspace{1cm}
		{\scshape\LARGE Facultad de Matem\'atica y Computaci\'on \par}
		\vspace{3cm}
		{\scshape\Huge Estad\'isticas para conocer al a\~no \par}
		\vspace{1cm}
		{\bfseries\LARGE Proyecto Final de Modelos de Optimizaci\'on II \par}
		{\scshape\Large Equipo 6 \par}
		\vspace{3cm}
		\vfill
		{\Large \underline{Integrantes:} \par}
		{\Large Olivia Gonz\'alez Pe\~na C-411 \par}
		{\Large Sheyla Cruz Castro C-412 \par}
		{\Large Laura Brito Guerrero C-412 \par}
		{\Large Juan Carlos Casteleiro Wong C-411 \par}
		\vfill
	\end{titlepage}
	
	
	\section{Planteamiento del problema}
		\textbf{Usuario final: Fernando Rodr\'iguez Flores @fernan2rodriguez}
		\subsection{Objetivos}
			El objetivo de este proyecto es dise\~nar e implementar un sistema (¿de encuestas? ¿de preguntas? ¿de tarjetas por el d\'ia de las madres?) que permita obtener informaci\'on sobre un grupo de clases al comienzo del curso. Entre lo que se desea conocer est\'an los intereses de los estudiantes (qu\'e les gusta y a qu\'e le dedican su tiempo libre). Cu\'anto conocen sobre determinados temas (videojuegos, pel\'iculas, libros). Si juegan, leen o consumen series, y en esos casos qu\'e tipos de juegos, libros, o series. Por ejemplo, en caso de que a alguien le guste la poes\'ia, qu\'e tipo de poes\'ia, qu\'e autores. ¿Qu\'e tipo de m\'usica? Etc. ¿Cu\'al es su relaci\'on con las asignaturas que ya vieron? ¿Cu\'ales le gustaron? ¿Cu\'ales no les gustaron? ¿Cu\'ales les dio absolutamente lo mismo? ... etc. \\
			El objetivo de esta primera parte es determinar qu\'e tipo de elementos y referencias se pueden incluir en las conferencias y clases pr\'acticas de forma que sean relevantes para la mayor cantidad de personas posibles. \\
			Tambi\'en interesa conocer los niveles de creatividad y de sentido del humor que tengan los estudiantes. Esto es a trav\'es de los \textit{tests} y \textit{cuestionarios} ya establecidos para esos fines. \\
			Algunas de estas informaciones conviene que sea \textit{con nombre}, otras a lo mejor conviene que sea an\'onima. \\
			Como parte del proyecto de debe:
			\begin{enumerate}
				\item Crear un mecanismo que permita recoger esta informaci\'on \textit{de la mejor manera posible}. \textit{De la mejor manera posible} significa que quiz\'as la mejor opci\'on no sea ponerles una encuesta con varios cientos de incisos ... o a lo mejor s\'i.
				\item Crear un sistema que permita analizar la informaci\'on recogida.
				\item Mostrar la informaci\'on recopilada de una manera que resulte \'util para la planificaci\'on de las asignaturas.
			\end{enumerate}
			
	\section{Modelo de soluci\'on}
		\subsection{Implementaci\'on del Backend}
			\textbf{Responsable: Laura Brito Guerrero} \\
			En el dise\~no de la l\'ogica del problema se necesita la implementaci\'on de diversas estructuras de datos las cuales nos permite el manejo de la informaci\'on m\'as facil para con el conocimiento de las mismas aplicar los algoritmos correspondientes que llevan a la soluci\'on del problema planteado. 
			La idea central a seguir se basa en el an\'alisis estad\'istico de las respuestas de los estudiantes a las encuestas, tales an\'alisis se realizan tanto de manera individual como general. Para dar un mayor entendimiento se pasa a explicar detalladamente la implementaci\'on de la misma.
			\subsubsection{poll.py}
				En este fichero se recogen los datos de la aplicaci\'on, es decir, a los estudiantes, los temas a considerar en las encuestas y las respuestas de los estudiantes a las mismas. Dicha informaci\'on se recoge en forma de diccionario. Se tiene en consideraci\'on que las encuestas pueden ser tanto an\'onimas o no, esto no afecta a los algoritmos dise\~nados. \\
				En el m\'etodo \textbf{analize} se recoge esta informaci\'on y conforme a esto se crean las estructuras de datos correspondientes, primeramente se dise\~na un contexto de temas (\textbf{Theme\underline{ }Context}), el cual ordena eficientemente los temas siendo estos a la vez subtemas de otros temas, esto se dise\~na jer\'arquicamente. \\
				A la vez, los estudiantes se recopilan en \textbf{Student\underline{ }Preferences}, donde se guardan los datos de los estudiantes en caso preciso y las respuestas de los mismos a las encuestas. \\
				Se analizan los estudiantes en pares no repetidos, se tiene la certeza que si se compara el estudiante \textit{i} con el estudiante \textit{j} se obtienen los mismos resultados que si se analiza al estudiante \textit{j} con el \textit{i}. Se comparan los criterios de preferencia de los estudiantes, qu\'e tan alejados o cercanos est\'an los mismos en sus gustos. Antes de explicar el proceso de comparaci\'on se pasa a explicar la organizaci\'on de las preferencias de cada estudiante. \\
				Una vez que se recojan las respuestas de cada estudiante (\textit{Student\underline{ }Preferences}) en \textbf{answers} (diccionario con llave \textit{tema} y valor correspondiente: \textit{respuesta del estudiante en ese tema}), se pasa a analizar sus preferencias (\textbf{preferences}) en orden descendente. \\
				Se implementa una clase \textbf{AVLSearchBinaryTree\underline{ }Insert} la cual crea un AVL con valores en los nodos (\textit{tema}, \textit{cantidad de respuestas al tema}), mientras mayor cantidad de aciertos tenga ese tema (que se considera una hoja de la jerarqu\'ia de los temas, ya que los temas \textit{padres} se van desglosando en subtemas \textit{hijos}, los cuales a su vez tienen su propia jerarqu\'ia, hasta caer en un subtema \textit{hoja}, en la cual los estudiantes desarrollan o \textit{marcan} sus gustos correspondientes), mayor es su preferencia. Estos nodos se organizan por el segundo valor de la tupla en forma ascendente. Luego de estar confeccionado el AVL, se realiza un recorrido entre-orden y devuelve una lista ordenada de las preferencias y las mismas se guardan en orden inverso en \textbf{preferences}. \\
				Como por cada estudiante se tiene un orden en sus preferencias, se analizan los pares de estudiantes, esto se implementa en \textbf{invert\underline{ }count}. Se toman \textbf{s1.prefences} y \textbf{s2.preferences} (\textbf{s1} y \textbf{s2} estudiantes) y se comparan, donde sin p\'erdida de generalidad se considera como arreglo ordenado a \textbf{s1.preferences}. Entonces, se ordena \textbf{s2.preferences} respecto a \textbf{s1.preferences}, y dependiendo de la cantidad de inversiones que tenga $tema_{i2}$ (el tema \textit{i} de \textbf{s2}) con respecto a $tema_{i1}$ (tema \textit{i} de \textbf{s1}) se establece la lejan\'ia o cercan\'ia del $tema_{i}$ que presentan ambos estudiantes. El criterio de cercan\'ia (\textbf{s.likes}) o lejan\'ia (\textbf{s.dislikes}), lo brinda un porcentaje, si la cantidad de inversiones es menor que un \textit{p} porciento (no se define porque cada tema o encuesta tiene su propio patr\'on definido) entonces el $tema_{i}$ es com\'un entre ambos estudiantes en su orden de preferencias. En caso contrario, los gustos de ambos estudiantes con respecto al tema est\'an alejados. Despu\'es de conocer la relaci\'on entre ambos estudiantes se a\~nade a las estructuras de datos correspondientes en la instancia de \textbf{s1} y \textbf{s2}. \\
				Luego de recorrer y analizar los gustos de todos los pares de estudiantes, se recurre al an\'alisis general estad\'istico, el cual se guarda en la instancia de \textbf{Theme\underline{ }Context},  y se implementa en el m\'etodo \textbf{theme\underline{ }context.stadistics\underline{ }result}. \label{stadisticsMethod}
			\subsubsection{hierarchy.py} \label{hierarchySec}
				En dicho fichero se implementa la clase \textbf{Theme\underline{ }Context} y \textbf{Theme\underline{ }Scope}. En \textbf{Theme\underline{ }Scope} se encuentra la jerarqu\'ia de los temas, y en \textbf{Theme\underline{ }Context} se guardan estos temas y se realizan los an\'alisis referentes a los mismos, como por ejemplo el an\'alisis estad\'istico con respecto a las respuestas de los estudiantes en \ref{stadisticsMethod}.
			\subsubsection{likes\underline{ }dislikes.py}
				Se encuentra la clase \textbf{Student\underline{ }Preferences}, donde se guardan los datos de cada estudiante, sus respuestas a las encuestas, la relaci\'on que guarda este con los \textit{n - 1} estudiantes restantes, sus preferencias con respecto a los temas.
			\subsubsection{invert\underline{ }array.py}
				Se realiza el an\'alisis \textbf{invert\underline{ }count} mencionado en \ref{stadisticsMethod}.
			\subsubsection{stadistics.py}
				Se implementa la clase \textbf{Stadistics\underline{ }Op}, la cual se encarga de realizar el an\'alisis estad\'istico por tema. Se implementa el c\'alculo de las medidas de Tendencia Central y las medidas de Dispersi\'on, necesarias para una correcta interpretaci\'on. Dicho an\'alisis por tema se instancia en \ref{hierarchySec}, donde se guarda en el diccionario \textbf{stadistics} (\textit{tema} : \textit{an\'alisis estad\'istico}).
\end{document}